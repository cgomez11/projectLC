\documentclass[fleqn,usenatbib]{mnras}
%=========================================================================
\usepackage{amsmath}
\usepackage{amssymb}
\usepackage{graphicx}
\usepackage{grffile}
\usepackage{float}
\usepackage[dvips]{epsfig}
\usepackage{epsfig}
\usepackage{dblfloatfix}
\usepackage{color}
\usepackage{caption}
\usepackage{hyperref}
\usepackage{bm}
\usepackage[british]{babel}
%Non reposionated tables
\newcommand{\HI}{{\text{H\MakeUppercase{\romannumeral 1}}} }
\newcommand{\lya}{\ifmmode{{\rm Ly}\alpha}\else Ly$\alpha$\ \fi}
\newcommand{\kms}{\ifmmode\mathrm{km\ s}^{-1}\else km s$^{-1}$\fi}
\newcommand{\vrot}{\ifmmode v_{\mathrm{rot}}\else $v_{\mathrm{rot}}$~\fi}
\newcommand{\vout}{\ifmmode v_{\mathrm{out}}\else $v_{\mathrm{out}}$~\fi}
\newcommand{\tauh}{\ifmmode \tau_{\mathrm{H}}\else $\tau_{\mathrm{H}}$~\fi}
\newcommand{\vth}{\ifmmode v_{\mathrm{th}}\else $v_{\mathrm{th}}$~\fi}
\newcommand{\hatk}{\ifmmode \hat{k}\else $\hat{k}$~\fi}
\newcommand{\STD}{\ifmmode \mathrm{STD}\else $\mathrm{STD}$~\fi}
\newcommand{\SKW}{\ifmmode \mathrm{SKW}\else $\mathrm{SKW}$~\fi}
\newcommand{\BI}{\ifmmode \mathrm{BI}\else $\mathrm{BI}$~\fi}

\begin{document}

%=========================================================================
%		FRONT MATTER
%=========================================================================
\title[The Local Group in Simulations]{Defining the Local Group in
  Cosmological Simulations}  

\author[Catalina G\'omez et al.]{
  Cataline G\'omez$^{1}$
  \thanks{c.gomez10@uniandes.edu.co},
  Jaime E. Forero-Romero $^{2}$
  \&
  More People $^{3}$
  \\
  %%
  $^{1}$ Departamento de Ingenier\'ia Biom\'edica, Universidad de los
  Andes, Cra. 1 No. 18A-12, CP 111711, Bogot\'a, Colombia \\
  $^{2}$ Departamento de F\'isica, Universidad de los Andes, Cra. 1
  No. 18A-10 Edificio Ip, CP 111711, Bogot\'a, Colombia \\
  $^{3}$ Somewhere\\
}

\maketitle

\begin{abstract}

 \end{abstract}


\section{Introduction}
 
\section{Related Work}
Previous studies have found a range of estimates of the individual masses of the MW and M31, or the total mass of the LG\cite{fattahi2016apostle}, limiting the straightforward/direct comparison between simulations and the observed Local Group. The selection of viable Local Group candidates from simulations relies on the kinematics of the LG members. Below there is a review of the literature, and the values of selection criteria used in each scenario. 
%We are not the 99 percent: quantifying asphericity in the distribution of Local Group satellites: https://arxiv.org/pdf/1805.03188.pdf
In the recent work of \cite{forero2018we}, they proposed a characterization of the global satellite distribution ranked by different selection criteria.

%APOSTLE: https://arxiv.org/pdf/1507.03643.pdf
Based on the match with kinematics of the LG members, \cite{fattahi2016apostle} found the constraints on the mass of the LG and compared it with estimates from other methods. The candidates were selected from the Millennium Simulations (I and II) and used for resimulations at different resolution levels. The reference conditions for the MW-M31 pair were defined as follows: the separation distance between the two galaxies is $787 \pm 25$ $kpc$, the approaching velocity (relative radial velocity) is $123 \pm 4$ $kms^{-1}$, the tangential velocity is $7$ $kms^{-1}$. Additionally, there is an isolation environment around \cite{forero2018we}
the pair: there are no galaxies brighter than the Large Magellanic Cloud within 3 Mpc from the MW. 

The first constraint is applied on the separation distance between the pair members, keeping the ones separated by a distance within the range $[600kpc-1.0Mpc]$. Next, the virial mass of each pair member (that fulfill the previous constraint) must not exceed $10^{11} M_\odot$, and the pair mass must be less than $10^{13} M_\odot$. These pairs must be in an isolated medium, which means that the cannot be other halo more massive than the less massive member of the pair within $2.5Mpc$ from the center of the pair. They evaluated different values for this threshold distance, making the isolation more or less restricted. To visualize the total mass distribution of the pairs that satisfy the previous conditions, they included three new criteria that can be imposed individually or combined. These are the following: (i) the relative radial velocity is within the range $[-175,75]$ $kms^{-1}$; (ii) the tangential velocity within the range $[0,50]$ $kms^{-1}$, and (iii) the Hubble flow velocity is in the observed range. Only 14 pairs satisfied the velocity criteria simultaneously. Furthermore, from these 14 pairs, the authors selected a subsample for the resimulation using more flexible constraints, and analyzed the the properties (brightness, mass, velocities) of the satellite population around each pair.

An important finding from the selection procedure proposed by \cite{fattahi2016apostle} was that the relative radial velocity constraint favors high-mass pairs $(5\times 10^{12} M_\odot)$, while the tangential velocity constraint low-mass pairs $(6\times 10^{11} M_\odot)$


%The mass distribution and gravitational potential of the MW:https://arxiv.org/pdf/1608.00971.pdf

%THE KINEMATICS OF THE LOCAL GROUP IN A COSMOLOGICAL CONTEXT: https://arxiv.org/pdf/1303.2690.pdf

%THE LOCAL GROUP IN THE COSMIC WEB: https://arxiv.org/pdf/1408.3166.pdf


%The dark matter assembly of the Local Group in constrained cosmological simulations of a ΛCDM universe: https://arxiv.org/pdf/1107.0017.pdf
%important figures: #2

%THE STELLAR MASS STRUCTURE OF MASSIVE GALAXIES FROM Z = 0 TO Z = 2.5; SURFACE DENSITY PROFILES AND HALF-MASS RADII: https://arxiv.org/pdf/1208.4363.pdf



\section{Methods}
\textit{Catalogue Selection}

 


\bibliographystyle{mnras}
\bibliography{references}


\end{document}
